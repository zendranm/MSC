\chapter{Abstract}
Deepfake is a technique from image-to-image translation class of problems, designed to combine and overlay objects in images or videos creating deceptively realistic counterfeits. This paper analyzes and compares four leading methods used for deepfake generation: autoencoders, variational autoencoders,  variational autoencoders generative adversarial networks and cycle generative adversarial networks, in problem of face-to-face conversion. Due to the lack of numerical methods for deepfake comparison, all obtained results were assessed in a visual evaluation process. Variational autoencoders technique has proved to be the most efficient one in facial-deepfake generation problem. The worst results were obtained from CycleGAN method, which proved to be unfitted for geometric changes and shape transformations. It was concluded that VAE-GAN technique is the one with the greatest potential, as in case of feature maps with better quality, this approach could provide deceivingly resembling deepfakes.